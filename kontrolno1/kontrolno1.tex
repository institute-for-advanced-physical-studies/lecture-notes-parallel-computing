% Copyright 2019 Clara Eleonore Pavillet

% Author: Clara Eleonore Pavillet
% Description: This is an unofficial Oxford University Beamer Template I made from scratch. Feel free to use it, modify it, share it.
% Version: 1.0

\documentclass{beamer}
\usepackage{pdfpages}
\input{Packages.tex}
\usetheme{oxonian}
\usepackage{wrapfig}
\usepackage{verbatim}
\usepackage{listings}

\title{OpenMP. Контролна работа 1.}
\subtitle{\textit{Курс „Паралелно програмиране“}}
%\titlegraphic{{\includegraphics[width=5.3cm]{iaps.png}}} 

\author{\newline \newline Стоян Мишев}

\vspace{1cm}

\date{} %\today

\begin{document}
\lstset{language=Python}
{\setbeamertemplate{footline}{} 
\frame{\titlepage}}


%%%%%%%%%%%%%%%%%%%%%%%%%%%%%%%%%%%%%%%%%%%%%%%%%%%

\begin{frame}[plain]
  \frametitle{Задача}

  Нека имаме два вектора \texttt{a} и \texttt{b} с равен брой елементи \texttt{N}>$10^3$. Запълнете векторите с реални числа.

  Използвайки OpenMP намерете:

  \begin{itemize}
  \item сумата от елементите на \texttt{a} и \texttt{b};
  \item тензорното произведение \texttt{C} на \texttt{a} и \texttt{b}:
    \begin{equation}
      C = \begin{bmatrix}
        a_1 b_1       & a_1 b_2  & a_1 b_3 & \dots & a_1 b_N \\
        a_2 b_1       & a_2 b_2  & a_2 b_3 & \dots & a_2 b_N \\
        \hdotsfor{5} \\
        a_N b_1       & a_N b_2  & a_N b_3 & \dots & a_N b_N \\
      \end{bmatrix};\nonumber
    \end{equation} 
  \item сумата от всички елементи на \texttt{C};
  \item колко нишки използвахте за изчисляването на сумата на елементите на \texttt{C}.
  \end{itemize}
В Moodle предавате source кода и изпълнимия (компилиран) файл. Крайният срок е 23:59 на 27.10.2025 г.
\end{frame}



\end{document}


%%% Local Variables:
%%% mode: latex
%%% TeX-master: t
%%% End:

