% Copyright 2019 Clara Eleonore Pavillet

% Author: Clara Eleonore Pavillet
% Description: This is an unofficial Oxford University Beamer Template I made from scratch. Feel free to use it, modify it, share it.
% Version: 1.0

\documentclass{beamer}
\usepackage{pdfpages}
\input{Packages.tex}
\usetheme{oxonian}
\usepackage{wrapfig}
\usepackage{listings}

\title{Използване на OpenMP - част 4. Tasks.}
\subtitle{\textit{Курс „Паралелно програмиране“}}
\titlegraphic{{\includegraphics[width=5.3cm]{iaps.png}}} 

\author{\newline \newline Стоян Мишев}

\vspace{1cm}

\date{} %\today

\begin{document}
\lstset{language=Python}
{\setbeamertemplate{footline}{} 
\frame{\titlepage}}


\section*{План}\begin{frame}{План}\tableofcontents\end{frame}

%%%%%%%%%%%%%%%%%%%%%%%%%%%%%%%%%%%%%

\begin{frame}
  \frametitle{Свързан списък. Задача.}
  \centering
  \includegraphics[width=0.3\textwidth]{while}
\end{frame}

\begin{frame}
  \frametitle{Свързан списък. Решение 1.}
  \centering
  \includegraphics[width=0.9\textwidth]{solution-naive}
\end{frame}

\begin{frame}
  \frametitle{Задачи в OpenMP(tasks)}
  Обособени единици за изпълнение със собствен код и данни, които се ``вземат'' от различни нишки чрез планировчик (scheduler - internal control variables).

  \centering
  \includegraphics[width=0.35\textwidth]{task}
\end{frame}


\begin{frame}[plain]
  \frametitle{Задачи в OpenMP. Fibonacci.}
  \centering
  \includegraphics[width=0.35\textwidth]{fib}\pause  
  \includegraphics[width=0.42\textwidth]{fib1}  
\end{frame}


\begin{frame}[plain]
  \frametitle{Задачи в OpenMP. Списък.}
  \centering
  \includegraphics[width=0.45\textwidth]{task-list}
\end{frame}

\begin{frame}[plain]
  \frametitle{Задачи в OpenMP. Списък.}
  \centering
  \includegraphics[width=0.55\textwidth]{task-list1}
  \includegraphics[width=0.65\textwidth]{task-list1-image}
\end{frame}


\begin{frame}
  \frametitle{Домашна работа}
  от \textit{Introduction to OpenMP 14 Module 8} до \textit{Introduction to OpenMP 17 Discussion 7}
\end{frame}

\end{document}


%%% Local Variables:
%%% mode: latex
%%% TeX-master: t
%%% End:

